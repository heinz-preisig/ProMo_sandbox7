
% automatically generated - do not edit
%
% =======================================

\documentclass[landscape, 12pt]{article}
\usepackage[T1]{fontenc}
\usepackage{longtable}
\usepackage[utf8]{inputenc}
\usepackage{graphicx}

\usepackage[text={\textwidth,10cm},
		a4paper,
		headsep=0.7cm,
		top=2cm,
		bottom=2cm,
		left=2cm,
		right=2cm,
		%		showframe,
		showcrop,
		%nofoot,
		nomarginpar
		]{geometry}
\usepackage[%
        backref,
        %	ps2pdf,
        %	dvips,
        final,
        %pdftitle={#1},
        pdfauthor={Heinz A Preisig},
        pdfsubject={'CAM'},
        colorlinks=true,
        linkcolor=red,
        citecolor=blue,
        anchorcolor=blue,
        hyperindex=true,
        bookmarks,
        bookmarksopen=true,
        bookmarksnumbered=true,
        bookmarksdepth=10,
        hyperfigures=true,
        %a4paper,
        linktocpage=true,
        pageanchor=true,
        ]{hyperref}

\usepackage{enumitem}

%% math packages
\usepackage{amsmath}
\usepackage{amssymb}
\usepackage{calligra}
\usepackage{array}

%% >>>>>>>>>>>>>>>> user definitions BIBLIOGRAPHY & DEFS <<<<<<<<<<<<<<<<<<<<<<<<

\input{./resources/defs}
\input{./resources/defvars}
\parindent=0pt




\newenvironment{eq}{\begin{minipage}{16cm}$}{$\end{minipage} }
%\newenvironment{eq}{$}{$}

% ==================================== body  =====================================
\begin{document}


\section*{Equation assignment sequence for variable $U^{e,circuit}$}



\renewcommand{\arraystretch}{1.5}

\begin{longtable}{|p{1cm}|p{1cm}|p{1cm}|p{16cm}|p{4cm}|}\hline
  no &var &equ &quations &token\\\hline\hline
\endhead
\hline \multicolumn{3}{r}{\textit{Continued on next page}} \\
\endfoot
\hline
\endlastfoot
9 &1 &- &\begin{eq}{\#}{_{}} :: \text{port variable}\end{eq} & \\
8 &170 &- &\begin{eq}{1}{_{N}} :: \text{port variable}\end{eq} & \\
7 &187 &198 &\begin{eq}{i}{_{}} := Root\left( {i}{_{N}}\right)\end{eq} & \\
6 &182 &178 &\begin{eq}{k^{e}}{_{N}} := {i}{_{N}} \, . \, \left( {U^{e}}{_{N}} \right)^{-1}\end{eq} & \\
5 &173 &197 &\begin{eq}{i}{_{N}} := {1}{_{N}} \, . \, {i}{_{}}\end{eq} & \\
4 &173 &196 &\begin{eq}{i}{_{N}} := {k^{e}}{_{N}} \, . \, {U^{e}}{_{N}}\end{eq} & \\
3 &160 &182 &\begin{eq}{U^{e}}{_{N}} := \left( {k^{e,\xi}}{_{N}} \right)^{-1} \, . \, {i}{_{N}}\end{eq} & \\
2 &160 &177 &\begin{eq}{U^{e}}{_{N}} := \text{Instantiate}({U^{e}}{_{N}}, {\#}{_{}})\end{eq} & \\
1 &181 &176 &\begin{eq}{U^{e,circuit}}{_{N}} := \text{Instantiate}({U^{e,circuit}}{_{N}}, {0}{_{}})\end{eq} & \\
0 &181 &175 &\begin{eq}{U^{e,circuit}}{_{N}} := {1}{_{N}} \, . \, {U^{e}}{_{N}}\end{eq} & \\

\end{longtable}

\end{document}
% ================================== end =========================================
